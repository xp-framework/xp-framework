% ********** Appendix 1 **********
\section{Installation}
\label{sec:app1:inst}

Die Installationsanweisungen in diesem Kapitel beziehen sich im Allgemeinen auf ein unixoides
System, sind aber leicht f"ur Windows anpassbar.

\subsection{Java}
\label{sec:app1:inst:java}

Zun"achst muss ein Java-SDK der Version 6.0 oder h"oher vorhanden sein, zu finden auf der
Java-Homepage \cite{JAVAHP}. Das Installationsverzeichnis des SDK wird im weiteren Verlauf als
\emph{JAVA\_HOME} bezeichnet.

\subsection{PHP}
\label{sec:app1:inst:php}
Weiterhin muss PHP aus den Quellen "ubersetzt werden. Hierzu muss zun"achst der PHP-Sourcecode in
einer Version 5.2.0 oder h"oher von der PHP-Homepage \cite{PHPHP} oder direkt aus dem PHP 
CVS-Repository heruntergeladen werden. Das Verzeichnis der PHP-Quellen wird im weiteren Verlauf
als \emph{PHP\_HOME} bezeichnet. PHP kann nun mittels des Befehls \emph{configure} konfiguriert werden,
der Anwender kann hier jeden gewohnten Parameter benutzen, wichtig ist nur dass der Paramter
\emph{enable-embed} auf \emph{shared} gesetzt wird. Nach dieser Konfiguration kann PHP mittels des
Befehls \emph{make} "ubersetzt werden:

\begin{lstlisting}[caption=Konfigurieren und "Ubersetzen von PHP]
~ # cd $PHP_HOME
php-5.2.0 # ./configure --enable-embed=shared 
php-5.2.0 # ./make
\end{lstlisting}

Im Verzeichnis \emph{PHP\_HOME/libs} sollte sich nun eine Bibliothek names \emph{libphp5.so} befinden.

\subsection{Turpitude}
\label{sec:app1:inst:turp}

Wechseln Sie nun in das Turpitude-Verzeichnis und "offnen Sie die Datei \emph{Makefile} mit einem 
Texteditor ihrer Wahl. "Andern sie die gekennzeichneten Zeilen ihrem System entsprechend:

\begin{lstlisting}[caption=Anpassen des Makefiles]
# vim Makefile
JAVA_HOME = /home/nsn/jdk1.6.0
PHP_HOME = /home/nsn/devel/php/php-5.2.0
\end{lstlisting}

Ein Aufruf des Befehls \emph{make} erzeugt nun zwei Dateien: \emph{turpitude.jar}, welches die
ben"otigten Java-Klassen enth"alt, sowie ein \emph{libturpitude.so}.

% ********** End of appendix **********
