% ********** Appendix 1 **********
\subsection{Hello World}
\label{sec:app1:hello}

Als erstes soll nun ein einfaches Programm entstehen, das mittels PHP die
Zeichenfolge "'Hello World!"' auf dem Bildschirm ausgibt. Dazu muss zun"achst
einmal eine Instanz der PHP-Scriptengine erzeugt werden:
\begin{lstlisting}[caption=Erzeugen der PHP-ScriptEngine]
ScriptEngineManager mgr = new ScriptEngineManager();
ScriptEngine eng = mgr.getEngineByName("turpitude");
\end{lstlisting}
Der ScriptEngineManager stellt Methoden zum Auffinden von ScriptEngines zur Verf"ugung,
die Methode \emph{getEngineByName()} gibt uns die gew"unschte Instanz zur"uck. Nun
kann das Script ausgef"uhrt werden:
\begin{lstlisting}[caption=Hello World Skript]
try {
    eng.eval("echo(\"Hello World!\n\");");
} catch(ScriptException e) {
    System.out.println("ScriptException caught:");
    e.printStackTrace();
}
\end{lstlisting}
Die Methode \emph{eval()} der ScriptEngine f"uhrt ein in einem String gespeichertes
Skript aus. Da es sich hierbei um Java-Quelltext handelt ist es n"otig etwaige Sonderzeichen
mit einem "'$backslash$"' zu maskieren.

Der Quelltext dieses Programmes befindet sich in der Datei \emph{HelloWorld.java}, und kann
mittels des Befehls \emph{make hello} ausgef"uhrt werden.

% ********** End of appendix **********
