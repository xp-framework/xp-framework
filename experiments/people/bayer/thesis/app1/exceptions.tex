% ********** Appendix 1 **********
\subsection{Java-Exceptions in PHP}
\label{sec:app1:exceptions}

Ein weiteres wichtiges Merkmal der Programmiersprache Java sind Exceptions. 
Turpitude bietet eine Reihe von Methoden die das Werfen und Fangen von Java-Exceptions
in PHP erlauben.

Zun"achst bringt das \emph{TurpitudeEnvironment} zwei Methoden mit die dem Anwender
das Werfen von Java-Exceptions erlauben: \emph{throw()} und \emph{throwNew()}. \emph{throw()}
erwartet als einzigen Parameter eine Instanz einer Java-Exception und wirft diese direkt,
w"ahrend \emph{throwNew()} zwei Parameter erwartet - den JNI-kodierten Klassennamen der
zu werfenden Exception, sowie die Nachricht (message) die diese Exception enthalten soll.
\begin{lstlisting}[caption=Werfen von Exceptions]
$class = $turpenv->findClass('java/lang/Exception');
$constructor = $class->findConstructor('(Ljava/lang/String;)V');
$instance = $class->create($constructor, 'Test');
$turpenv->throw($instance);
...
$turpenv->throwNew('java/lang/IllegalArgumentException', 'Test');
\end{lstlisting}
Um zu "uberpr"ufen, ob eine Exception aufgetreten ist kann die Methode \emph{exceptionOccurred()}
genutzt werden, welche die gerade aktuelle Exception zur"uckgibt:
\begin{lstlisting}[caption=Exceptions aufgetreten?]
if ($exc = $turpenv->exceptionOccurred()) {
    ...
}
\end{lstlisting}
Hat der Anwender die Fehlerbehandlung abgeschlossen muss die Methode \emph{exceptionClear()} aufgerufen
werden, um der JVM mitzuteilen dass sie die Exception als gefangen betrachten soll. Tut der Anwender
dies nicht gilt die Exception als nicht behandelt, und liegt weiterhin an:
\begin{lstlisting}[caption=Exceptions behandeln]
$turpenv->throwNew('java/lang/IllegalArgumentException', 'Test');
if ($exc = $turpenv->exceptionOccurred()) {
    printf(\"Msg: %s\\n\", $exc->toString('()Ljava/lang/String;'));
    $turpenv->exceptionClear();
}
\end{lstlisting}

Der Quelltext dieses Beispieles befindet sich in der Datei \emph{ExceptionSample.java}, 
und kann mittels des Befehls \emph{make exceptions} ausgef"uhrt werden.

% ********** End of appendix **********
