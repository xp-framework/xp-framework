% ********** Appendix 1 **********
\subsection{Aufrufen von PHP-Methoden}
\label{sec:app1:invoke}

Bisher wurde haupts"achlich besprochen wie ein PHP-Skript auf Java-Klassen, deren Attribute und
Methoden zugreiffen kann. In diesem Abschnitt soll der umgekehrte Weg gegangen, und aus Java heraus 
Methoden und Top-Level Funktionen eines PHP-Skriptes aufgerufen werden. Hierzu wird zun"chst
ein solches Skript geschrieben, in dem eine Klasse definiert wird, und das eine Funktion enth"alt die 
eine Instanz dieser Klasse zur"uckgibt:
\begin{lstlisting}[caption=PHP-Skript]
function useless($i) {
    return new foo($i);
}
 
class foo {
  var $val = '';
  function __construct($i) {
    $this->val = $i;
  }
  function bar($i) {
    return 'foo::bar::'.$i.' ('.$this->val.')';
  }
}
\end{lstlisting}
Um nun solche Skript-Funktionen aufzurufen definiert der JSR223 ein weiteres Interface, \emph{Invocable}.
Laut Spezifikation soll eigentlich lediglich die jeweilige \emph{ScriptEngine} dieses Interface implementieren,
allerdings w"are dann nicht klar auf welchem Skript die Funktion aufgerufen werden soll. Deswegen implementiert
das \emph{PHPCompiledScript} dieses Interface ebenfalls, und die \emph{PHPScriptEngine} leitet die Interface-Aufrufe
an das zuletzt "ubersetzte Skript weiter. Zun"achst muss das Skript also "ubersetzt werden, siehe hierzu auch
Kapitel \ref{sec:app1:invoke}.
\begin{lstlisting}[caption="Ubersetzen]
ScriptEngine eng = mgr.getEngineByName("turpitude");
Compilable comp = (Compilable)eng;
CompiledScript script = comp.compile(/*Source*/);
\end{lstlisting}
Nachdem das Skript "ubersetzt ist kann die globale Funktion aufgerufen werden:
\begin{lstlisting}[caption=Funktionsaufruf]
Invocable inv = (Invocable)script;
Object phpobj = inv.invokeFunction("useless", "Function Value");
\end{lstlisting}
Das phpobj h"alt nun eine Referenz auf die Instanz der PHP-Klasse \emph{foo}. Jetzt kann bei dieser
die Methode \emph{bar()} aufgerufen werden:
\begin{lstlisting}[caption=Methodenaufruf]
Object retval = inv.invokeMethod(phpobj, "bar", "Method Value");
\end{lstlisting}
Die Methoden \emph{invokeFunction()} und \emph{invokeMethod()} haben sehr "ahnliche Signaturen, beide
erwarten den Namen der Funktion/Methode als String und die zu "ubergebenden Argumente als weitere Parameter,
\emph{invokeMethod()} erwartet zus"atzlich ein PHPObjekt auf dem die Methode aufgerufen werden soll.

Ein weiteres Feature des \emph{Invocable}-Interfaces ist die Methode \emph{getInterface()}, die in zwei
Ausf"uhrungen existiert. Diese Methode gibt eine in der jeweiligen Skriptsprache implementierte Instanz eines
"ubergebenen Interfaces zur"uck, auf die dann in Java wie auf ein "'echtes"' Java-Objekt zugegriffen werden kann.
Zur Demonstation ein simples Java-Interface:
\begin{lstlisting}[caption=Java-Interface]
public interface ExampleInterface {
    public String bar(String s);
}
\end{lstlisting}
Zuf"alligerweise implementiert die PHP-Klasse \emph{foo} bereits dieses Interface, und da wir dies
wissen, k"onnen wir das phpobj \emph{getInterface()} "ubergeben, und diese Methode gibt uns ein 
entsprechendes Java-Objekt zur"uck, auf dem wir ganz normal arbeiten k"onnen:
\begin{lstlisting}[caption=getInterface() mit "ubergebenem Objekt]
ExampleInterface exint;
exint = inv.getInterface(phpobj, ExampleInterface.class);
exint.bar("interface call");
\end{lstlisting}
Falls kein passendes Objekt zur Verf"ugung steht kann der Anwender die Methode \emph{getInterface()}
auch ohne den ersten Parameter aufrufen, dann versucht Turpitude anhand des simplen Klassennamens
(bei \emph{java.lang.String} w"are dieser \emph{String}) eine passende Klasse zu finden, und mittels
des Default-Konstruktors eine Instanz dieser Klasse zu erzeugen. Erweitern wir als das obige
PHP-Skript um folgende Zeilen:
\begin{lstlisting}[caption=Interface Implementation in PHP]
class ExampleInterface {
  function bar($i) {
    return 'ExampleInterface::bar::'.$i;
  }
}
\end{lstlisting}
K"onnen wir einfach folgenden Java-Code benutzen um die Methode \emph{bar()} aufzurufen,
diesmal bei der PHP-Klasse \emph{ExampleInterface}:
\begin{lstlisting}[caption=getInterface() mit "ubergebenes Objekt]
exint = inv.getInterface(ExampleInterface.class);
exint.bar("interface call");
\end{lstlisting}


Der Quelltext dieses Beispieles befindet sich in der Datei \emph{InvocableSample.java}, 
und kann mittels des Befehls \emph{make objects} ausgef"uhrt werden. Der Quelltext des
verwendeten Beispielinterfaces befindet sich in der Datei \emph{ExampleInterface.java}.

% ********** End of appendix **********
