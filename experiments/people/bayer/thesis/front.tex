% *************** Front matter ***************

% ***************************************************
% You should specify the contents of title page here
% Then you can specify dedication page or disable it
% ***************************************************

% *************** Title page ***************
\pagestyle{empty}
\sffamily

\noindent
\begin{center}
    \Large
    Hochschule f"ur Technik und Wirtschaft Karlsruhe\\
    Fakult"at f"ur Informatik\\
\end{center}

\vfill\vfill
\begin{center}
    \large
    Diplomarbeit\\
\end{center}

\vfill
\begin{center}
    \Huge\bfseries
    PHP in Java - ein Experiment
\end{center}

\vfill
\begin{center}
    \Large
    von 
\end{center}

\vfill
\begin{center}
    \huge\bfseries
    Michael Bayer
\end{center}

\vfill\vfill\vfill
\begin{center}
    \Large
    Betreuer: Timm Friebe,  Prof. Dr. Thomas Fuch\ss
\end{center}

\vfill
\begin{center}
\large
    Karlsruhe, 2007
\end{center}

\cleardoublepage

% *************** Declaration **************
\pagestyle{empty}
Ich erkl"are hiermit, dass ich die vorliegende Diplomarbeit 
selbst"andig erarbeitet und verfasst habe; aus fremden Quellen 
"ubernommene Gedanken sind als solche kenntlich gemacht. Die Arbeit 
wurde bisher keiner anderen Pr"ufungsbeh"orde vorgelegt. 

\vspace{1cm}
Karlsruhe, den 

\vspace{2cm}
\hfill Michael Bayer

\cleardoublepage


% *************** abstract
\chapter{Zusammenfassung}

Die Anforderungen an Softwaresysteme werden heute immer komplexer, gleichzeitig aber
werden Produktlebenszyklen und somit auch die Entwicklungszeit immer k"urzer. Deswegen
gewinnen diensteorientierte Architekturen (SOA) immer mehr an Bedeutung, da sie eine
Wiederverwendung bereits entwickelter Komponenten erlauben. Innerhalb der 1\&1-Firmengruppe
werden hierzu Technologien der \emph{Java Enterprise Edition} eingesetzt. F"ur Entwickler
anderer Programmiersprachen ist es daher wichtig, mit solchen Diensten kommunizieren
zu k"onnen. Die Abteilung \emph{i::Dev}, in welcher diese Diplomarbeit erstellt wurde,
setzt f"ur viele Zwecke die Skriptsprache PHP ein, welche keine M"oglichkeit bietet
mit JavaEE-Diensten zu kommunizieren. Dies hat unter anderem zur Folge, dass
sehr viele bestehende Programme und Komponenten keine M"oglichkeit besitzen mit Java EE-Diensten
zu kommunizieren.

Im Laufe dieser Arbeit wird ein Softwaresystem erstellt, das es erm"oglicht
PHP-Skripte innerhalb einer Java Virtual Machine auszuf"uhren. Weiterhin
wird ein Datenaustausch zwischen den Laufzeitumgebungen der beiden 
Programmiersprachen, sowie der Zugriff auf Funktionen, Objekte und Methoden
der jeweils anderen Sprache erm"oglicht, mit dem Ziel Enterprise Java Beans in
PHP zu entwickeln und transparent in einem JavaEE Application Server zu installieren.

Zu diesem Zweck werden verschiedene Technologien zum Einbetten einer Skriptsprache
in Java evaluiert. Die so gewonnenen Erkenntnisse werden schlie\ss lich 
genutzt, um eine dieser Technologien auszuw"ahlen: den \emph{Java Specification Request 223}.
Im zweiten Teil der Arbeit wird eine JSR 223-Implementierung f"ur PHP erstellt, sowie
der PHP-Interpreter derart erweitert, dass er auf den vollen Funktionalit"atsumfang
Javas zugreifen kann. Im dritten Teil der Arbeit wird die erstellte Bibliothek genutzt,
um voll funktionsf"ahige Enterprise Java Beans in PHP zu realisieren. 
Hierzu werden EJB 3.0 Technologien eingesetzt,
und es wird er"ortert wie ein PHP-Anwender Stateful- und Stateless Session Beans sowie 
Message Driven Beans in seiner gewohnten Programmierumgebung entwickeln kann.

Die entwickelte Bibliothek wird unter dem Namen \emph{Turpitude} als Open Source-Projekt
ver"offentlicht.

\chapter{Abstract}

Enterprise software systems today face ever more complex requirements, while product
lifecycles and with them development timespans grow shorter. Therefore the importance of
service-oriented architectures grows rapidly, because they allow the re-use of services and
software-components. Whithin 1\&1 \emph{Java Enterprise Edition} technologies are used to
build such software-systems. It is important for developers of other languages to be able
to communicate with JavaEE-services. The department \emph{i::Dev} within this thesis was
written uses the scripting language PHP for many projects. PHP does not have the capabilities
to communicate with such services. This results in many existing programs and components lacking
the capablilities to communicate with JavaEE-services. 

This thesis investigates different technologies that allow embeding scripting languages into
a Java Virtual Machine. One of these technologies - the \emph{Java Specification Request 223} - 
is then used to implement a library that not only allows users to execute PHP-scripts within a
JVM, but also extends the PHP runtime environment to enable PHP-scripts to create and access Java classes, 
objects and methods. It further enables the Java-user to access PHP objects, methods and functions.
Ultimate goal of this thesis is to be able to write \emph{Enterprise Java Beans} in PHP and deploy
them transparently in a JavaEE Application Server.
The third part of this thesis then shows how the library can be used to implement EJBs in PHP, at
the same time evaluating the differences between EJB 3.0 and earlier specifications.

\cleardoublepage

% *************** Table of contents ***************
\pagenumbering{roman}
\pagestyle{headings}
\tableofcontents

% *************** End of front matter ***************
