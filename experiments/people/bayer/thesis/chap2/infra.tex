\section{Infrastruktur}
\label{sec:chap2:infra}

Bevor mit der Bearbeitung der Aufgabe begonnen werden konnte musste zun"achst die n"otige Infrastruktur
geschaffen werden, allem voran die Auswahl des zu verwendenden Application Servers. Diese Wahl war leicht
zu treffen, der bei 1\&1 ausschlie\ss lich eingesetzten Application Server ist JBoss \cite{JBOSSHP}.
Folglich wurde ein JBoss in der Version 4.0.5 installiert, wobei darauf zu achten war dass der
nicht standardm"a\ss ig mitinstallierte Deployer f"ur EJB 3.0 Applikationen zus"atzlich ausgew"ahlt wurde.
Direkt nach der Installation konnte der Application Server problemlos gestartet werden.

Nun wurde f"ur das Projekt eine Verzeichnisstruktur aufgebaut wie Sun sie f"ur EJB-Projekte vorschl"agt, mit
eigenen Unterverzeichnissen f"ur die Quelltexte (src/), ben"otigte Bibliotheken (lib/), zus"atzliche 
Deploymentdeskriotoren (dd/) und die beim Buildprozess erzeugten Dateien (build/).
Da es sich um ein reines Java-Projekt handelte wurde als Buildsystem Apache Ant \cite{ANTHP} gew"ahlt.


