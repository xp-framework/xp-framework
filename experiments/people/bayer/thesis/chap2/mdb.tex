\section{Message Driven Beans}
\label{sec:chap2:mdb}

Eine Message Driven Beans ist eine EJB die kein eigentliches Service-Interface implementiert,
und deren Methoden somit nicht direkt aufgerufen werden k"onnen. Vielmehr "'h"ort"' sie auf
Nachrichten in einer JMS Message Queue und bearbeitet diese.  JMS - der Java Messaging Service - 
ist ein Teil des Application Servers, und erlaubt das Senden von Nachrichten (Messages) in
Queues, an welche sich Dienste anh"angen k"onnen um auf diese Nachrichten zu reagieren,
auf diese Art und Weise wird die Realisierung asynchroner Abl"aufe m"oglich.
In fr"uheren EJB Versionen mussten diese Queues umst"andlich "uber XML-Dateien konfiguriert werden,
mit EJB 3.0 werden sie dynamisch erzeugt sobald ein Zuh"orer sich an eine Queue h"angt, oder
sobald ein Sender eine Nachricht in eine Queue schickt. Das Entwickeln einer Message Driven Bean 
gestaltet sich denkbar einfach, es muss lediglich das Interface \emph{MessageListener} implementiert
werden, welches eine einzige Methode \emph{onMessage()} definiert.


