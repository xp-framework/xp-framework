% ********** Chapter 1 **********
\chapter{Fazit und Ausblick}
\label{sec:extro}

Aufgabe dieser Diplomarbeit war es PHP in die Welt der Java Enterprise Edition Application Server
einzuf"uhren, zum einen um eine einfachere Kommunikation zwischen Anwendungen der beiden Programmiersprachen
zu erm"oglichen, und zum anderen um dem PHP-Anwender zu erlauben die Vorteile die ein solcher Application Server
und die Java an sich bietet in vollem Umfang auszunutzen. Hierzu wurde zun"achst evaluiert welche M"oglichkeiten
"uberhaupt existieren eine Skriptsprache in Java einzubetten, und welche existierenden Technologien hierzu
genutzt werden k"onnten. Insbesondere wurden zwei Technologien beleuchtet, das Bean Scripting Framework
vom Apache Jakarta Projekt, und der Java Specification Request 223. Letzterer wurde schliesslich auch 
auserw"ahlt um die Aufgabe zu l"osen.

In Kapitel \ref{sec:chap1} wurde erl"autert wie eine Bibliothek namens Turpitude erstellt wurde, die den JSR 223 implementiert,
und so die Ausf"uhrung von PHP-Skripten innerhalb einer Java-Anwendung, sowie den Zugriff auf Objekte, Funktionen
und Methoden in diesen Skrioten erlaubt. Dar"uberhinaus erm"oglicht diese Bibliothek dem PHP-Entwickler den
Zugriff auf die komplette Funktionalit"at der Java-Umgebung, sowie den Datenaustausch zwischen den beiden
Programmiersprachen. Objekte der jeweils anderen Sprache werden nicht als blo\ss e Kopien der enthaltenen Daten,
sondern als echte Referenzen vorgehalten, was einen transparenten Umgang mit ihnen in der Wirtsprogrammiersprache
erlaubt. Diese M"oglichkeiten machen Turpitude einzigartig, es existiert kein Softwarepaket das "ahnliche
M"oglichkeiten er"offnet. Dieser Teil der Aufgabe nahm einen ganz erheblichen Teil der zur Verf"ugung stehenden
Zeit in Anspruch, weswegen entschieden wurde bei den verbleibenden Teilen an Umfang zu k"urzen.

Kapitel \ref{sec:chap2} wird schliesslich ein m"ogliches Einsatzszenario f"ur Turpitude erforscht: Das Schreiben von
Enterprise Java Beans in PHP. Es wurden erfolgreich sowohl Stateful- als auch Stateless Session Beans und Message Driven Beans
in PHP geschrieben und deployt. Die verbliebene Zeit reichte nicht aus einen Automatismus zu entwickeln, der
dem PHP-Entwickler das Schreiben von Java-Quelltext komplett abnimmt, allerdings wurden die L"osungen so gew"ahlt, dass
die Implementierung eines solchen weiterhin m"oglich bleibt. Es wurde somit gezeigt dass sich Turpitude durchaus eignet
um in den Systemen die in der Abteilung und der Firma ben"otigt werden eingesetzt zu werden.

Zusammenfassend kann die Aufgabenstellung als vollst"andig gel"ost gewertet werden, zwar wurde sie an einigen Stellen
gek"urzt, aber nur um an anderen Stellen eine Vertiefung zu erlauben.
Noch muss sich Turpitude zwar im echten Produktiveinsatz beweisen, der Autor hat diesbez"uglich aber keine Bedenken.
An der Bibliothek selbst sind noch eine Vielzahl von Verbesserungen m"oglich, sowohl was die Qualit"at des Quelltextes
als auch was den Umfang der gebotenen Funktionalit"at angeht. Beispielsweise k"onnte der Array-Zugriff in PHP auf alle
Java-Objekte ausgedehnt werden, die das Interface Iteratable implementieren. Ausserdem w"are eine weitere Vereinfachung
des Zugriffs auf Methoden und Attribute w"unschenswert, mit dem Ziel den Benutzer komplett von der Notwendigkeit die
JNI-Kodierungen zu kennen zu befreien. Auch bessere Kontrolle "uber die Typisierung beim Methodenaufruf und beim Zugriff
auf Attribute w"are w"unschenswert.

Es besteht sowohl innerhalb als auch au\ss erhalb von 1\&1 ein gro\ss es Interesse an Turpitude, und es sind viele Bereiche
abseits der Web- und Enterprise-Welt denkbar in denen die Bibliothek sinnvoll eingesetzt werden k"onnte, ein Beipiel hierf"ur
w"aren ein Plugin f"ur die Entwicklungsumgebung Eclipse das eine bessere Integration von PHP bietet und so ein komfortables
Entwickeln von PHP-Skripten mit Eclipse erlauben w"urde. Dem kommt zugute, dass Turpitude unter einer Open Source Lizenz ver"offentlicht
werden wird, um die Bibliothek eine m"oglichst weite Verbreitung erfahren zu lassen.
Die M"oglichkeiten die Turpitude er"offnet sind schier endlos, und der Autor freut sich auf Projekte die mit 
Hilfe der Bibliothek erstellt werden.


% ********** End of chapter **********
