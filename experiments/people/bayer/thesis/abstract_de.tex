\chapter{Zusammenfassung}

Die Anforderungen an Softwaresysteme werden heute immer komplexer, gleichzeitig aber
werden Produktlebenszyklen und somit auch die Entwicklungszeit immer k"urzer. Deswegen
gewinnen diensteorientierte Architekturen (SOA) immer mehr an Bedeutung, da sie eine
Wiederverwendung bereits entwickelter Komponenten erlauben. Innerhalb der 1\&1-Firmengruppe
werden hierzu Technologien der \emph{Java Enterprise Edition} eingesetzt. F"ur Entwickler
anderer Programmiersprachen ist es daher wichtig, mit solchen Diensten kommunizieren
zu k"onnen. Die Abteilung \emph{i::Dev}, in welcher diese Diplomarbeit erstellt wurde,
setzt f"ur viele Zwecke die Skriptsprache PHP ein, welche keine M"oglichkeit bietet
mit JavaEE-Diensten zu kommunizieren. Dies hat unter anderem zur Folge, dass
sehr viele bestehende Programme und Komponenten keine M"oglichkeit besitzen mit Java EE-Diensten
zu kommunizieren.

Im Laufe dieser Arbeit wird ein Softwaresystem erstellt, das es erm"oglicht
PHP-Skripte innerhalb einer Java Virtual Machine auszuf"uhren. Weiterhin
wird ein Datenaustausch zwischen den Laufzeitumgebungen der beiden 
Programmiersprachen, sowie der Zugriff auf Funktionen, Objekte und Methoden
der jeweils anderen Sprache erm"oglicht, mit dem Ziel Enterprise Java Beans in
PHP zu entwickeln und transparent in einem JavaEE Application Server zu installieren.

Zu diesem Zweck werden verschiedene Technologien zum Einbetten einer Skriptsprache
in Java evaluiert. Die so gewonnenen Erkenntnisse werden schlie\ss lich 
genutzt, um eine dieser Technologien auszuw"ahlen: den \emph{Java Specification Request 223}.
Im zweiten Teil der Arbeit wird eine JSR 223-Implementierung f"ur PHP erstellt, sowie
der PHP-Interpreter derart erweitert, dass er auf den vollen Funktionalit"atsumfang
Javas zugreifen kann. Im dritten Teil der Arbeit wird die erstellte Bibliothek genutzt,
um voll funktionsf"ahige Enterprise Java Beans in PHP zu realisieren. 
Hierzu werden EJB 3.0 Technologien eingesetzt,
und es wird er"ortert wie ein PHP-Anwender Stateful- und Stateless Session Beans sowie 
Message Driven Beans in seiner gewohnten Programmierumgebung entwickeln kann.

Die entwickelte Bibliothek wird unter dem Namen \emph{Turpitude} als Open Source-Projekt
ver"offentlicht.
