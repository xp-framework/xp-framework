
\section{TurpitudeEnvironment}

Die Klasse \texttt{TurpitudeEnvironment} ist Basis aller Java-Zugriffe aus PHP heraus.
Sie kann entweder jederzeit wie jede normale PHP-Klasse per \texttt{new} erzeugt werden,
eine Instanz wird beim Start des Interpreters in das Superglobal \texttt{\$\_SERVER}
injiziert. Sie bietet folgende Methoden:

%\hline
\begin{itemize}
\item
\texttt{TurpitudeJavaClass findClass(string classname)}\\
Diese Methode erstellt aus einem String, der einen Klassennamen enth"alt, ein
Klassenobjekt. Der Klassenname muss JNI-kodiert sein. Siehe hierzu \cite{JNIHP}.
\item
\texttt{void throwNew(string classname, string message)}\\
Wirft eine Java-Exception des "ubergebenen Typs mit der "ubergebenen Nachricht.
\item
\texttt{void throw(TurpitudeJavaObject$<$java.lang.Throwable$>$ ex)}\\
Wirft die "ubergebene Exception. \texttt{ex} muss vom Typ \texttt{TurpitudeJavaObject}
sein, und ein Java-Objekt des Typs \texttt{java.lang.Throwable} enthalten.
\item
1. \texttt{bool instanceOf(TurpitudeJavaObject a, TurpitudeJavaClass b)}\\
2. \texttt{bool instanceOf(TurpitudeJavaObject a, String b)}\\
Gibt \texttt{true} zur"uck, wenn \texttt{a} ein Java-Objekt ist, und bei 1. eine Instanz
der in \texttt{b} repr"asentierten Java-Klasse b, bei 2. eine Instanz der im String \texttt{b} 
beschriebenen Klasse ist.
\item
\texttt{TurpitudeJavaObject exceptionOccurred()}\\
Gibt die anliegende Java-Exception zur"uck. Liegt keine Exception vor, gibt die Methode
\texttt{null} zur"uck.
\item
\texttt{void exceptionClear()}\\
L"oscht alle anliegenden Java-Exceptions.
\item
\texttt{TurpitudeJavaArray newArray(string type, int len)}\\
Erzeugt ein Java-Array des gew"unschten Typs, mit der gew"unschten L"ange.
\item
\texttt{TurpitudeJavaObject$<$ScriptContext$>$ getScriptContext()}\\
Gibt den ScriptContext der ScriptEngine zur"uck.
\end{itemize}
