
\chapter{Abstract}

Enterprise software systems today face ever more complex requirements, while product
lifecycles and with them development timespans grow shorter. Therefore the importance of
service-oriented architectures grows rapidly, because they allow the re-use of services and
software-components. Whithin 1\&1 \emph{Java Enterprise Edition} technologies are used to
build such software-systems. It is important for developers of other languages to be able
to communicate with JavaEE-services. The department \emph{i::Dev} within this thesis was
written uses the scripting language PHP for many projects. PHP does not have the capabilities
to communicate with such services. This results in many existing programs and components lacking
the capablilities to communicate with JavaEE-services. 

This thesis investigates different technologies that allow embeding scripting languages into
a Java Virtual Machine. One of these technologies - the \emph{Java Specification Request 223} - 
is then used to implement a library that not only allows users to execute PHP-scripts within a
JVM, but also extends the PHP runtime environment to enable PHP-scripts to create and access Java classes, 
objects and methods. It further enables the Java-user to access PHP objects, methods and functions.
Ultimate goal of this thesis is to be able to write \emph{Enterprise Java Beans} in PHP and deploy
them transparently in a JavaEE Application Server.
The third part of this thesis then shows how the library can be used to implement EJBs in PHP, at
the same time evaluating the differences between EJB 3.0 and earlier specifications.
