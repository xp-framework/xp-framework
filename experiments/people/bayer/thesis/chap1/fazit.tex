% ********** Chapter 1 **********
\section{Fazit}
\label{sec:chap1:fazit}

Im Laufe dieses Kapitels wurde eine m"achtige Bibliothek entwickelt, die nicht nur das Ausf"uhren von
PHP-Skripten aus einer Java-Laufzeitumgebung heraus erm"oglicht, sondern viel weiter geht und die beiden
Programmiersprachen eng miteinander verbindet. Dem PHP-Entwickler wird die M"oglichkeit geboten den
vollen Funktionsumfang von Java inklusive aller verf"ugbaren Java-Bibliotheken zu nutzen, Objekte zu erzeugen,
auf deren Felder und Methoden zuzugreifen und diese auf intuitive und einfache Weise in PHP zu nutzen.
Der Java-Entwickler kann PHP-Skripte nicht nur ausf"uhren, ihnen Daten "ubergeben und von ihnen erzeugte
Daten zur"uckerhalten, er kann nicht nur gezielt Funktionen und Methoden innerhalb eines PHP-Skriptes
aufrufen, sondern er kann sogar Java-Interfaces direkt in PHP implementieren, oder bereits bestehende PHP-Klassen
in ein Java-Interface wrappen und transparent mit ihnen wie mit jedem anderen Java-Objekt umgehen.

Dieser Funktionsumfang macht Turpitude einzigartig. Es existieren zwar zwei "'Konkurrenzprodukte"' auf dem
Markt - die JSR223-Implementierung von Zend und die PHP/Java-Bridge \cite{BRIDGEHP} - allerdings implementieren
beide weder den vollen Umfang des JSR223, noch halten sie die aktuelle Spezifikation ein. Weiterhin ist die
Zend-Implementierung nicht Queloffen, sprich es ist einem Anwender nicht m"oglich ein eigenes PHP in Java
zu benutzen, mit allen Extensions die er ben"otigt, und es ist ebenfalls unm"oglich sie auf nicht explizit
unterst"utzten Plattformen einzusetzen. Die PHP/Java-Bridge ist zwar im Quelltext verf"ugbar, aber sie
benutzt eine TCP-Verbindung um mit einer laufenden Java-Umgebung zu kommunizieren. Diese Kommunikation
basiert au\ss erdem noch auf XML, die Nachteile einer solchen Kommunikation wurden in Kapitel \ref{sec:background}
hinl"anglich beschrieben. Die Tatsache, dass der verwendete PHP-Interpreter aus einer externen Bibliothek
geladen wird stellt sicher, dass der Anwender jegliches PHP-Programm ausf"uhren kann. Insbesondere
wird so der komplette Funktionsumfang des XP-Frameworks unterst"utzt.

Somit wurden alle am Anfang des Kapitels gesteckten Ziele nicht nur erreicht sondern in vielen F"allen sogar
"ubererf"ullt.

Die Implementierung der Bibliothek selbst wurde haupts"achlich durch zwei Faktoren erschwert:
Zum einen ist die JSR 223-Spezifikation in vielen Detailfragen nur "au\ss erst ungenau und erscheint au\ss erdem
in einigen Bereichen etwas undurchdacht. Dies lies sich allerdings in den meisten F"allen durch eine
angemessene Portion Pragmatismus sehr schnell und zur Zufriedenheit Aller l"osen. Deutlich mehr Zeit
kostete die mangelhafte Dokumentation der Zend-Engine. Viele Probleme konnten nur durch langwieriges
Ausprobieren und das Lesen des Zend-Quelltextes gel"ost werden, was oft zu unn"otigen Verz"ogerungen 
des Projektes f"uhrte. Abgesehen von den veralteten und nicht gepflegten Dokumentationsbruchst"ucken
auf der PHP-Seite (\cite{PHPHP}) war die einzig wirklich hilfreiche "'Dokumentation"' die, die durch das Programm
\emph{LXR} (siehe \cite{LXRHP}) erzeugt wird und unter \cite{PHPLXR} verf"ugbar ist. Der Quelltext
von PHP selbst stellte weitere Herausforderungen, so ist er nur sehr unzureichend kommentiert, und verwendet
einige eher "'interessante"' Konzepte. Das verdeutlichen am besten einige Beispiele, zum einen die 
Implementierung einer verketteten Liste:

\begin{lstlisting}[caption=Verkettete Liste im Zend-Code]
typedef struct _zend_llist_element {
    struct _zend_llist_element *next;
    struct _zend_llist_element *prev;
    char data[1]; /* Needs to always be last in the struct */
} zend_llist_element;
...
zend_llist_element *le;
...
opline_ptr = (zend_op *)le->data;
\end{lstlisting}

Hier wird genau ein Byte Speicher reserviert, es werden allerdings Pointer an diese Stelle geschrieben,
die mindestens vier Bytes lang sind - ein Voidpointer w"are hier passender gewesen.
Zum anderen noch ein Ausschnitt aus der De-serialisierungsroutine. Hier wurden zwar Kommentare
verwendet, allerdings zeugen sie an dieser Stelle nicht unbedingt von einer guten Organisation:

\begin{lstlisting}[caption=Zend-Engine: De-serialisierung]
yy13:   ++YYCURSOR;
    goto yy14;
yy14:
{
    /* this is the case where we have less data than planned */
    php_error_docref(
        NULL TSRMLS_CC, 
        E_NOTICE, 
        "Unexpected end of serialized data");
    return 0; /* not sure if it should be 0 or 1 here? */
}
\end{lstlisting}

Trotz dieser Probleme konnte ein benutzbares Softwaresystem erstellt werden, auch wenn Turpitude sicherlich
noch keinen Produktionsstatus erreicht hat, hierzu muss der Quelltext noch hinreichend auf Speicherlecks
"uberpr"uft werden, da der Einsatz in einem Web- oder Application-Server immer lange Laufzeiten mit sich bringt,
und Speicherlecks in diesen F"allen besonders gravierend sind. Weiterhin kann an vielen Stellen unn"otig gewordener
Quelltext entfernt oder "ahnliche Quelltextstellen zu eigenen Funktionen refaktoriert werden. Die Begrenzte Zeit
die f"ur diese Arbeit zur Verf"ugung stand lie\ss\ solche "'Sch"onheitsoperationen"' aber leider nicht zu.

Nach Einsch"atzung des Autors kann Turpitude zu diesem Zeitpunkt aber schon guten Gewissens eingesetzt werden um 
nicht unternehmenskritische Anwendungen zu entwickeln, vor allem weil eventuelle qualitative Verbesserungen
sich nicht mehr auf die Schnittstellen auswirken werden. Es gilt allerdings zu beachten, dass die Bibliothek unter
keinen Umst"anden genutzt werden sollte um nicht vertrauensw"urdige PHP-Skripte auszuf"uhren, zum einen weil
nicht alle Grenzf"alle ausgetestet wurden, und zum anderen weil durch den Zugriff auf Java-Klassen die PHP-eigenen
Schutzmechanismen ausgehebelt werden k"onnten. Ein solcher Einsatz beispielsweise um Kunden auf einem Java-Webserver 
PHP anzubieten w"are - zumindest zu diesem Zeitpunkt - grob fahrl"assig.

Im Laufe der Implementierung wurde deutlich, dass eine - im Gegensatz zur urspr"unglichen Planung - vollst"andige
Implementierung des JSR223 und vor allem die Implementierung des Zugriffs auf Java-Funktionalit"at aus dem PHP-Interpreter heraus
dem Unternehmen und der Abteilung viele Vorteile bringen w"urde, weswegen
deutlich mehr Zeit in die Entwicklung der Bibliothek investiert wurde als eigentlich geplant. Dies f"uhrte nun
dazu dass f"ur den Rest der Aufgabe deutlich weniger Zeit zur Verf"ugung stand, weswegen in diesen Bereichen an
Umfang gek"urzt werden musste.

% ********** End of chapter **********
