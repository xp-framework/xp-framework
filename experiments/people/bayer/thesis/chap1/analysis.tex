% ********** Chapter 1 **********
\section{Analyse}
\label{sec:chap1:ana}
Dieses Kapitel erl"autert wie der erste Teil der Aufgabenstellung - das Ausf"uhren von PHP Sourcecode innerhalb
einer Java-Anwendung - erf"ullt wurde, welche Technologien heirzu benutzt, aber auch welche nicht benutzt wurden.
Die Anforderung war m"oglichst grosse Teile des XP-Frameworks innerhalb von Java auszuf"uhren.
Vom vollen Umfang des XP-Frameworks ausgenommen waren lediglich die Komponenten welche eine Kommunikation
nach aussen erlaubten, so zum Beispiel die Datenbankkonnektivit"at, da diese Funktionen sp"ater vom Application Server
bereitgestellt werden sollten. Idealerweise sollte die verwendete PHP Version leicht austauschbar sein. Eine weitere
Anforderung war die Interaktion von Java nach PHP und umgekehrt m"oglichst einfach zu gestalten, Java-Objekte sollen
in PHP erzeug- und zugreifbar sein.

Wichtige Use-Cases f"ur diesen Teil des Projektes sind:
\begin{enumerate}
\item Test, ob die PHP-ScriptEngine verf"ugbar ist. Durch einfache Ausgabe auf der Konsole soll schnell ermittelbar sein, welche
    ScriptEngines geladen und einsatzbereit sind. Dies dient haupts"chlich der "Uberpr"ufung, ob die aktuelle Java-Umgebung
    richtig konfiguriert ist.
\item Test, ob die PHP-ScriptEngine funktioniert. Es soll ein einfaches PHP-Skript ausgef"uhrt werden, um dem Anwender
    die korrekte Funktionsweise der ScriptEngine vorzuf"uhren und zu pr"ufen ob die nativen Teile der Implementation
    ordnungsgem"a\ss funktionieren.
\item Ausf"uhren beliebiger PHP-Skripte. Es soll dem Anwender m"oglich sein, ohne selbst Java-Quelltext zu schreiben, beliebige
    PHP-Skripte auszuf"uhren.
\item "Ubersetzen eines beliebigen Skriptes. Gem"a\ss des in \emph{javax.script} vorhandenen Interfaces \emph{Compilable} soll 
    ein beliebiges Skript "ubersetzt und mehrfach ausgef"uhrt werden k"onnen.
\item Aufrufen von Methoden in "ubersetzten Skripten. Das zu entwickelnde Softwaresystem soll das Aufrufen von Funktionen und
    Methoden in "ubersetzten Skripten erlauben. Hierzu soll das Interface \emph{javax.script.Invocable} unterst"utzt werden.
\item "Ubergabe von Daten mittels des Kontextes. Der Anwender soll Daten mittels des Engine-Kontextes "ubergeben und in PHP
    weiterverwenden k"onnen. Genauso soll es m"oglich sein Daten aus PHP heraus an die JVM zur"uckzu"ubergeben.
    Zur"uckgegebene Objekte
\item Setzen von php.ini Parametern. Dem Anwender soll die M"oglichkeit geboten werden eine alternative php.ini Datei zu verwenden,
    um das Laufzeitverhalten von PHP wie gewohnt zu beeinflussen.
\item Erzeugen von Java-Objekten in PHP. Dem Anwender soll erm"oglicht werden Java-Objekte zu erzeugen, auf deren
    Attribute zuzugreifen und deren Methoden aufzurufen. Java-Objekte sollen in PHP m"oglichst nicht nur als Kopien zug"anglich sein,
    sondern sollen viel mehr echte Referenzen sein.
\end{enumerate}
% ********** End of chapter **********
